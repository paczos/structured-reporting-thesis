\documentclass[12pt, twoside, openany]{report}
\usepackage[dvips]{graphicx,color,rotating}
\usepackage{a4wide}
\usepackage[utf8]{inputenc}
\usepackage{enumerate}
\usepackage{verbatim}
\usepackage[polish,british]{babel}
\usepackage[T1]{fontenc}
\usepackage{geometry}
\geometry{left=25mm,right=25mm,%
bindingoffset=10mm, top=25mm, bottom=25mm}
\usepackage{latexsym}
\usepackage{amsthm}
\usepackage{palatino}
\usepackage{array}
\usepackage{pstricks}
\usepackage{textcomp}
\theoremstyle{definition}
\newcommand*{\norm}[1]{\left\Vert{#1}\right\Vert}
\newcommand*{\abs}[1]{\left\vert{#1}\right\vert}
\newcommand*{\om}{\omega}

\author{Paweł Paczuski}
\title{Tytuł pracy}

\begin{document}

% Zażółć gęślą jaźń.
\begin{titlepage}
\pagestyle{empty}

\noindent
\begin{Large}
\begin{table}[t]
\centering
\begin{tabular}[t]{lcr}
 \includegraphics[width=70pt,height=70pt]{PW} & POLITECHNIKA WARSZAWSKA & \includegraphics[width=70pt,height=70pt]{ELKA}\\
& WYDZIAŁ ELEKTRONIKI & \\
& I TECHNIK INFORMACYJNYCH &
\end{tabular}
\end{table}

% \vfill
\begin{center}ENGINEER DIPLOMA THESIS\end{center}
\begin{center}Computer Science\end{center}\end{Large}
% \vfill
\begin{center}
\Huge
\textbf{Structured reporting system}
\end{center}
% \vfill\vfill
\vfill
\begin{center}
\Large
Author:\\
\LARGE
Paweł Paczuski
\end{center}
\vfill
\begin{center}
\Large
Promotor: prof. dr hab. inż. Jan J. Mulawka
\end{center}
\vfill
\begin{center}
\Large
Warszawa, miesiąc rok
\end{center}
\newpage
\hfill
\begin{table}[b]
\centering
\begin{tabular}[t]{ccc}
............................................. & \hspace*{100pt} & .............................................\\
podpis promotora & \hspace*{100pt} & podpis autora
\end{tabular}
\end{table}


% \maketitle
\end{titlepage}
\thispagestyle{empty}
\newpage
\pagestyle{headings}
\setcounter{page}{1}
\hyphenation{Syl-ves-tra}
\hyphenation{Syl-ves-ter-a}
\begin{otherlanguage}{british}
\begin{abstract}
    streszczenie po angielsku
\end{abstract}
\end{otherlanguage}

\begin{otherlanguage}{polish}
\begin{abstract}
streszczenie po polsku
\end{abstract}
\end{otherlanguage}

%-----------Początek części zasadniczej-----------

\chapter{Introduction}
\section{Traditional workflow of a radiologist}


\section{Thesis scope and objectives}
\section{Description of contents}
\chapter{Short review of types of software used in healthcare}
\chapter{Description of the proposed solution}
\chapter{Examples of application}

\chapter{Conclusion}
\appendix {How to use the program}


%-----------Koniec części zasadniczej-----------

\begin{thebibliography}{11}
\bibitem[1]{B} Stanisław Białas, \emph{Macierze. Wybrane problemy}, AGH Uczelniane Wydawnictwa Naukowo-Dydaktyczne, Kraków, 2006.
\bibitem[2]{H} Nicholas J. Higham, \emph{Accuracy and stability of numerical algorithms}, SIAM, Philadelphia 1996.
\bibitem[3]{H2} Nicholas J. Higham, \emph{Functions of Matrices. Theory and Computation}, SIAM, Philadelphia 2008.
\bibitem[4]{DJ} Maksymilian Dryja, Janina i Michał Jankowscy, \emph{Przegląd metod i algorytmów numerycznych, część 2}, Wydawnictwa Naukowo-Techiczne, Warszawa 1982.
\end{thebibliography}
\tableofcontents
\clearpage
\begin{otherlanguage}{polish}
\pagestyle{empty}
\noindent Warszawa, dnia ...............
\vspace{5cm}
\begin{center}
\LARGE{Oświadczenie}
\end{center}
Oświadczam, że pracę magisterską pod tytułem: ,,Tytuł pracy'', której promotorem jest prof. dr hab. Jan Wybitny, wykonałem/am samodzielnie, co poświadczam własnoręcznym podpisem.
\vspace{2cm}
\begin{flushright}
...........................................
\end{flushright}
\end{otherlanguage}
\end{document}